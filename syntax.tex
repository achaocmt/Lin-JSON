%!TEX root = draft.tex

\section{Syntax of JSON}
\label{sec:syntax of json}


\subsection{Syntax}
\label{subsec:syntax}

We define a simple command language that is executed locally at each replica, and could query and modify the local copies. The syntax of our command language contains three kinds of commands: %given $x,y \in \vars$, $\aval \in \vals$ and $i \in \mathbb{N}$,

\begin{itemize}
\setlength{\itemsep}{0.5pt}
\item[-] $x = u$: This command is used to assign a ``tree location'' to variable $x$. Here $x \in \vars$, and there are four possibilities of $u$:

    \begin{itemize}
    \setlength{\itemsep}{0.5pt}
    \item[-] $u = \mathit{doc}$,

    \item[-] $u = y.\mathit{key}$: Here $y \in \vars$ and $\mathit{key} \in \strings$,

    \item[-] $u = (\mathit{doc},\mathit{Reg}).\mathit{key}$, here $\mathit{key} \in \strings$,

    \item[-] $u = (o,t)$, Here $o \in \objids$ and $t \in \{ \mathit{Reg}, \mathit{Map},\mathit{List} \}$.
    \end{itemize}

\item[-] $u.m(a)\Rightarrow b$: This command represents that ``the object of $u$'' calls a method with some arguments. Here $m \in \methods$, $a,b,\ldots \in \datadomain$, and there are five possibilities for $u$:

    \begin{itemize}
    \setlength{\itemsep}{0.5pt}
    \item[-] $u = x$ with $x \in \vars$,

    \item[-] $u = \mathit{doc}$,

    \item[-] $u = \mathit{Parent}(x)$ with $x \in \vars$,

    \item[-] $u = (o,t)$ with $o \in \objids$ and $t \in \{ \mathit{Reg}, \mathit{Map},\mathit{List} \}$,

    \item[-] $u = \mathit{sib}(x)$ with $x \in \vars$.
    \end{itemize}

\item[-] ${\tt checkType}(x)\Rightarrow t$, where $x \in \vars$ and $t \in \{ \mathit{Reg}, \mathit{Map},\mathit{List} \}$. This command is used to return the type of $x$.

${\tt checkType}(\mathit{Parent}(x))\Rightarrow t$, which returns the type of ``parent of $x$''.

\item[-] ${\tt checkSibType}(x) \Rightarrow S_1$ (resp., ${\tt checkSibType}(\textit{doc}) \Rightarrow S_1$), where $x \in \vars$ and $S_1 \subseteq \{ \mathit{Reg}, \mathit{Map},\mathit{List} \}$. This command is used to return the type of $x$, as well as the type of ``other object in the same location of x'' (resp., the type of all object ``at the location of $\textit{doc}$'').
\end{itemize}



\subsection{Translation}
\label{subsec:translation}

With our commands, we could translate the original command of JSON as follows.


\begin{itemize}
\setlength{\itemsep}{0.5pt}
\item[-] $x = \mathit{doc}$: $x = \mathit{doc}$,

\item[-] $x = y.{\tt get}(\mathit{key}_1)$: $x = y.\mathit{key}_1$,

\item[-] $x = y.{\tt idx}(i)$: $(y.{\tt readList}()\Rightarrow a_1 \cdot \ldots \cdot a_k) \cdot (x = a_i)$,

\item[-] $x = \mathit{doc}.{\tt get}(\mathit{key})$: $x = (\mathit{doc},\mathit{Reg}).\mathit{key}$,

\item[-] $x = \mathit{doc}.{\tt idx}(i)$: $(\mathit{doc}.{\tt readList}()\Rightarrow a_1 \cdot \ldots \cdot a_k) \cdot (x = a_i)$,

\item[-] $u = \aval$ with $u \in \vars \cup \{ \mathit{doc} \}$, $\aval \neq \{\}$ and $\aval \neq []$: $({\tt checkSibType}(x) \Rightarrow S_1) \wedge ( \mathit{Reg} \in S_1 )$, then,

    \begin{itemize}
    \setlength{\itemsep}{0.5pt}
    \item[-] If $\mathit{Map} \in S_1$, then do $( \mathit{sib}(x).{\tt readMap}()\Rightarrow S ) \cdot ( \mathit{sib}(x).{\tt clearMap}(S) )$,

    \item[-] If $\mathit{List} \in S_1$, then do $( \mathit{sib}(x).{\tt readList}()\Rightarrow l ) \cdot ( \mathit{sib}(x).{\tt clearList}(set \ of \ l) )$,

    \item[-] Do $x.{\tt writeReg}(v,\mathit{id})$ for a new identifier $\mathit{id}$. Moreover, if ${\tt checkType}(\mathit{Parent}(x))\Rightarrow \mathit{Map}$, then we do $\mathit{Parent}(x).{\tt addMap}(\mathit{key}(x),x)$. 
    \end{itemize}

\item[-] $u = \aval$ with $u \in \vars \cup \{ \mathit{doc} \}$ and $\aval = \{\}$: $({\tt checkSibType}(x) \Rightarrow S_1) \wedge ( \mathit{Reg} \in S_1 )$, then,

    \begin{itemize}
    \setlength{\itemsep}{0.5pt}
    \item[-] If $\mathit{Reg} \in S_1$, then do $( \mathit{sib}(x).{\tt readRegIds}()\Rightarrow S ) \cdot ( \mathit{sib}(x).{\tt clearReg}(S) )$,

    \item[-] If $\mathit{List} \in S_1$, then do $( \mathit{sib}(x).{\tt readList}()\Rightarrow l ) \cdot ( \mathit{sib}(x).{\tt clearList}(set \ of \ l) )$,

    \item[-] Do $( x.{\tt readMap}()\Rightarrow S ) \cdot ( x.{\tt clearMap}(S) )$.
    \end{itemize}

\item[-] $u = \aval$ with $u \in \vars \cup \{ \mathit{doc} \}$ and $\aval = []$: $({\tt checkSibType}(x) \Rightarrow S_1) \wedge ( \mathit{Reg} \in S_1 )$, then,

    \begin{itemize}
    \setlength{\itemsep}{0.5pt}
    \item[-] If $\mathit{Reg} \in S_1$, then do $( \mathit{sib}(x).{\tt readRegIds}()\Rightarrow S ) \cdot ( \mathit{sib}(x).{\tt clearReg}(S) )$,

    \item[-] If $\mathit{Map} \in S_1$, then do $( \mathit{sib}(x).{\tt readMap}()\Rightarrow S ) \cdot ( \mathit{sib}(x).{\tt clearMap}(S) )$,

    \item[-] Do $( x.{\tt readList}()\Rightarrow l ) \cdot ( x.{\tt clearList}(set \ of \ l) )$,
    \end{itemize}

\item[-] $x.{\tt addAfter}(\aval)$ with $\aval \neq \{\}$ and $\aval \neq []$: $( \mathit{Parent}(x).{\tt addAfter}(x, ( \mathit{oid},\mathit{Reg} )) ) \cdot ( ( \mathit{oid},\mathit{Reg} ).$ ${\tt writeReg}( v,\mathit{id} ) )$ for new identifier $\mathit{oid}$ and $\mathit{id}$.

\item[-] $x.{\tt addAfter}(\aval)$ with $\aval = \{\}$: $( \mathit{Parent}(x).{\tt addAfter}(x, ( \mathit{oid},\mathit{Map} )) ) \cdot ( ( \mathit{oid},\mathit{Map} ).{\tt readMap}()\Rightarrow S ) \cdot ( ( \mathit{oid},\mathit{Map} ).{\tt clearMap}(S) )$ for new identifier $\mathit{oid}$.

\item[-] $x.{\tt addAfter}(\aval)$ with $\aval = []$: $( \mathit{Parent}(x).{\tt addAfter}(x, ( \mathit{oid},\mathit{List} )) ) \cdot ( ( \mathit{oid},\mathit{List} ).{\tt readList}\Rightarrow l ) \cdot ( ( \mathit{oid},\mathit{List} ).{\tt clearList}(set \ of \ l) )$ for new identifier $\mathit{oid}$.

\item[-] $x.{\tt delete}()$: First do $\mathit{checkType}(\mathit{Parent}(x))\Rightarrow t$ and we require that $t \in \{ \mathit{Map}, \mathit{List} \}$. Then,

    \begin{itemize}
    \setlength{\itemsep}{0.5pt}
    \item[-] If $t = \mathit{Map}$, then do $\mathit{Parent}(x).{\tt removeMap}( \mathit{key}(x) )$,

    \item[-] Else, if $t = \mathit{List}$, then do $\mathit{Parent}(x).{\tt removeList}(x)$.
    \end{itemize}

\item[-] $x.\mathit{mapValues}$: $x.{\tt readMapValues}()\Rightarrow S$,

\item[-] $x.\mathit{mapKeys}$: $x.{\tt readMapKeys}()\Rightarrow S$,

\item[-] $x.\mathit{listLength}$: $x.{\tt listLength}()\Rightarrow i$,

\item[-] $x.\mathit{regValues}$: $x.{\tt readRegValues}()\Rightarrow S$.
\end{itemize}


\forget{
\begin{itemize}
\setlength{\itemsep}{0.5pt}
\item[-] let $\avar = EXPR$, where $\avar \in \vars$,

\item[-] $EXPR = v$, where $v \in VAL$,

\item[-] $EXPR.addAfter(v)$, where $EXPR$ is an element of a list and $v \in VAL$,

\item[-] $EXPR.delete$, where $EXPR$ is an element of a list or an element of a map,

\item[-] $yield$,

\item[-] $EXPR.create(key,type)$, where $EXPR$ is a map,

\item[-] $EXPR.keys$, where $EXPR$ is a map,

\item[-] $EXPR.length$, where $EXPR$ is a list,

\item[-] $EXPR.values$, where $EXPR$ is a register,

\item[-] $EXPR.type$,

\item[-] $CMD; CMD$,
\end{itemize}

A \emph{cursor} $EXPR$ essentially points to an object of JSON, and such object has type. $EXPR$ is recursively defined as follows:

\begin{itemize}
\setlength{\itemsep}{0.5pt}
\item[-] $doc$,

\item[-] $\avar$, where $\avar \in \vars$,

\item[-] $EXPR.get(key)$, where $EXPR$ is a map,

\item[-] $EXPR.idx(i)$, where $EXPR$ is a list and $i \in \mathbb{N}$.
\end{itemize}
%A variable $x \in VAR$ is used to represent a shortcut of a cursor. Each variable $x$ must be initialized and it can be initialized only once.
}
