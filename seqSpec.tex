%!TEX root = draft.tex

\section{The Sequential Specification of JSON}
\label{sec:the sequential specification of JSON}


\subsection{Sequential Specification $\mathit{Spec}(\mathit{JSON})$}
\label{subsec:sequential specification Spec(JSON)}

A object id is one of the following cases:

\begin{itemize}
\setlength{\itemsep}{0.5pt}
\item[-] $(\mathit{doc},t)$ with $t \in \{ \mathit{Reg}, \mathit{Map}, \mathit{List} \}$,

\item[-] $(\mathit{doc},\mathit{Map}) \cdot (\mathit{key}_1,\mathit{Map}) \cdot \ldots \cdot (\mathit{key}_i,\mathit{Map}) \cdot (\mathit{key}_{\mathit{i+1}},t)$, where each $\mathit{key}_j$ is a string and $t \in \{ \mathit{Reg}, \mathit{Map}, \mathit{List} \}$,

\item[-] $(\mathit{id},t)$ with $t \in \{ \mathit{Reg}, \mathit{Map}, \mathit{List} \}$, where $\mathit{id}$ is a unique identifier,

\item[-] $(\mathit{id},\mathit{Map}) \cdot (\mathit{key}_1,\mathit{Map}) \cdot \ldots \cdot (\mathit{key}_i,\mathit{Map}) \cdot (\mathit{key}_{\mathit{i+1}},t)$, where $\mathit{id}$ is a unique identifier, each $\mathit{key}_j$ is a string and $t \in \{ \mathit{Reg}, \mathit{Map}, \mathit{List} \}$,
\end{itemize}

A candidate object id is either a object id, or $l \cdot a$, if $l \cdot (a,t)$ is a object identifier.

Each abstract state $\abstate = (X,O,Par,Key,f,S)$, where

\begin{itemize}
\setlength{\itemsep}{0.5pt}
\item[-] $X \subseteq \vars$ is a set of variables,

\item[-] $O$ is a set of object id,

\item[-] $Par$ is a function that maps each $x \in \vars$ into $y \in \vars$ or an element of $\{ (\textit{doc},\mathit{Reg}),(\textit{doc},\mathit{Map})$, $(\textit{doc},\mathit{List}) \}$. %$Par(x) = y$, if $y$ is a map or list object, and $x$ is a element of $y$,

\item[-] $Key$ is a function that maps each $x \in \vars$ into an element of $\strings$,

\item[-] $f$ is a function that maps each $x \in \vars$ into a candidate object id,

\item[-] $S$ is a function that maps each object id into its corresponding abstract state.
\end{itemize}

The transition relation of $\mathit{Spec}(\mathit{JSON})$ is defined as follows:

\noindent {\bf Transitions of the First Kind}:

$(X,O,Par,Key,f,S) \specarrow{ x = u } (X \cup \{x\}, O, Par[x \leftarrow p(u)], Key[x \leftarrow k(u)], f[x \leftarrow q(u)], S)$, if $x \notin X$.

$p(u)$ is defined as follows:

\begin{itemize}
\setlength{\itemsep}{0.5pt}
\item[-] $p(u) = y$, if $u=y.\mathit{key}$,

\item[-] $p(u) = (\mathit{doc},\mathit{Reg})$, if $u=(\mathit{doc},\mathit{Reg}).\mathit{Key}$,

\item[-] $p(u) = \bot$, otherwise.
\end{itemize}

$k(u)$ is defined as follows:

\begin{itemize}
\setlength{\itemsep}{0.5pt}
\item[-] $k(u) = \mathit{key}_1$, if $u = y.\mathit{key}_1$ or $u = (\mathit{doc},\mathit{Reg}).\mathit{key}_1$,

\item[-] $k(u) = \bot$, otherwise.
\end{itemize}

$q(u)$ is defined as follows:

\begin{itemize}
\setlength{\itemsep}{0.5pt}
\item[-] $q(u) = f(y).\mathit{key}_1$, if $u = y.\mathit{key}_1$,

\item[-] $q(u) = u$, otherwise.
\end{itemize}



\noindent {\bf Transitions of the Second Kind}:

$(X,O,Par,Key,f,S) \specarrow{ u.m(a)\Rightarrow b } (X, O \cup \{ o(u,m) \}, Par, Key, f[l(u) \leftarrow o(u,m)], S')$.

$l(u)$ is defined as follows:

\begin{itemize}
\setlength{\itemsep}{0.5pt}
\item[-] $l(u) = y$, if $u = y$,

\item[-] $l(u) = \bot$, otherwise.
\end{itemize}

$t(m)$ is defined as follows:

\begin{itemize}
\setlength{\itemsep}{0.5pt}
\item[-] $t(m) = \mathit{Reg}$, if $m \in \{ {\tt writeReg}, {\tt readRegValues}, {\tt readRegId}, {\tt clearReg} \}$,

\item[-] $t(m) = \mathit{Map}$, if $m \in \{ {\tt readMap}, {\tt clearMap}, {\tt readMapValues}, {\tt readMapKeys}, {\tt removeMap}, {\tt addMap} \}$,

\item[-] $t(m) = \mathit{List}$, if $m \in \{ {\tt readList}, {\tt clearList}, {\tt addAfter}, {\tt readListLength}, {\tt removeList} \}$.
\end{itemize}

$o(u,m)$ is defined as follows:

\begin{itemize}
\setlength{\itemsep}{0.5pt}
\item[-] $o(u,m) = l \cdot (c,t)$, if $u = x$, $f(x) = l \cdot (c,t)$ and $t=t(m)$,

\item[-] $o(u,m) = l \cdot (\mathit{key_1},t(m))$, if $u = x$ and $f(x) = l \cdot \mathit{key}_1$. Here $\mathit{key}_1$ is a string,

\item[-] $o(u,m) = (\mathit{doc},t(m))$, if $u = \mathit{doc}$,

\item[-] $o(u,m) = f(y)$, if $u = \mathit{Parent}(x)$, $\mathit{Par}(x) = y$, $f(y) = l \cdot (c,t)$ and $t = t(m)$,

\item[-] $o(u,m) = (\mathit{id},\mathit{Reg})$, if $u = (\mathit{id},\mathit{Reg})$ and $t(m) = \mathit{Reg}$,

\item[-] $o(u,m) = l \cdot (\mathit{key}_1,t(m))$, if $u = \mathit{sib}(x)$ and $f(x) = l \cdot \mathit{key}_1$,

\item[-] $o(u,m) = l \cdot (c,t(m))$, if $u = \mathit{sib}(x)$ and $f(x) = l \cdot (c,\_)$.
\end{itemize}

Let $g(a)$ be defined as follows:

\begin{itemize}
\setlength{\itemsep}{0.5pt}
\item[-] $g(a) = f(x)$, if $a = x$,

\item[-] $g(a) = Key(x)$, if $a = \mathit{key}(x)$,

\item[-] $g(a) = a$, otherwise.
\end{itemize}

$S'$ is a function that has the same value as $S$ everywhere, except for $o(u,m)$, and it is defined as follows:

\begin{itemize}
\setlength{\itemsep}{0.5pt}
\item[-] If $S[o(u,m)] = \abstate_{\bot}$, then $S'[o(u,m)] = \abstate$, where $\abstate_0 \specarrow{ m(g(a))\Rightarrow b } \abstate$. Here $\abstate_0$ is the initial abstract state of type $t(m)$,

\item[-] If $S[o(u,m)] \neq \abstate_{\bot}$, then $S'[o(u,m)] = \abstate$, where $S[o(u,m)] \specarrow{ m(g(a))\Rightarrow b } \abstate$.
\end{itemize}


\noindent {\bf Transitions of the Third Kind}:

$(X,O,Par,Key,f,S) \specarrow{ \mathit{checkType}(x)\Rightarrow t } (X, O, Par, Key, f, S)$, if $x \in X$. Here if $f(x) = l \cdot (\_,t_1)$, then $t = t_1$; Otherwise, $t = \bot$.

$(X,O,Par,Key,f,S) \specarrow{ \mathit{checkType}(\mathit{Parent}x)\Rightarrow t } (X, O, Par, Key, f, S)$, if $x \in X$. Here if $Par(x) = y$ and $f(x) = l \cdot (\_,t_1)$, then $t = t_1$; Otherwise, $t = \bot$.


\noindent {\bf Transitions of the Forth Kind}:

$(X,O,Par,Key,f,S) \specarrow{ \mathit{checkSibType}(x)\Rightarrow S_1 } (X, O, Par, Key, f, S)$, if $x \in X$. Here

\begin{itemize}
\setlength{\itemsep}{0.5pt}
\item[-] If $f(x) = l \cdot (c,t)$, then for each $t' \in \{ \mathit{Reg}, \mathit{Map}, \mathit{List} \}$, $t' \in S_1$, if $l \cdot (c,t') \in O$,

\item[-] If $f(x) = l \cdot \mathit{key}_1$, then for each $t' \in \{ \mathit{Reg}, \mathit{Map}, \mathit{List} \}$, $t' \in S_1$, if $l \cdot (\mathit{key}_1,t') \in O$.
\end{itemize}

$(X,O,Par,Key,f,S) \specarrow{ \mathit{checkSibType}(\mathit{doc})\Rightarrow S_1 } (X, O, Par, Key, f, S)$, if $x \in X$. Here for each $t' \in \{ \mathit{Reg}, \mathit{Map}, \mathit{List} \}$, $t' \in S_1$, if $(\mathit{doc},t') \in O$.





%In this section we give the sequential specification $\mathit{Spec}(\mathit{JSON})$ of JSON. To describe sequential specifications in a succinct way we will provide an operational description. To that end, we will associate to specifications a notion of abstract state, which we shall generally denote by $\abstate$ and its domain shall be denoted by $\abstates$. Then, to each valid label $\alabel$ we will associate a transition relation $\abstate \specarrow{\alabel}  \abstate'$ which, given an abstract state $\abstate$ and provided that the label $\alabel$ can be applied in $\abstate$, produces a new abstract state $\abstate'$. In the specific case where the label $\alabel$ assumes a certain precondition $\apre$ over the initial abstract state $\abstate$ we will use Hoare-style preconditions and write
%\(\big(\abstate\ |\ \apre(\abstate)\big) \specarrow{\alabel}
%  \abstate'\).
%In this way, a sequential specification is the set of labels that are accepted by the successive application of the transition relation starting from some given initial state $\abstate_{0}$.





\subsection{Sequential Specification of Multi-Value Register, Map and List}
\label{subsec:sequential specification of multi-value register, map and list} 

In this subsection we give sequential specification and abstract state of multi-value register, map and list. 


\noindent {\bf Specification of Multi-Value Register}: %We use the sequential specification of multi-value register of our previous work \cite{RA-lin}. 
Each abstract state $\abstate$ is a set of tuples $(a,id)$, where $a$ is a data and $id$ is an identifier. Moreover, we assume that there is a partial order on identifier. The sequential specification $\specMVReg$ of multi-value register is defined by:

\[
  \begin{array}{rcl}
    %\abstate
    %& \specarrow{\alabellong[\mathtt{readIds}]{}{\abstate}{}}
    %& \abstate\\
    \big(\ \abstate\ |\ \neg (\mathit{id} \leq \abstate) \big)
             & \specarrow{ \alabelshort[{\tt write}]{a,id} }
    & \abstate \cup \{ (a,\mathtt{id}) \} \setminus \{ (a',id') \in \abstate, id' < id \} \\
    \abstate
    & \specarrow{\alabellong[\mathtt{readRegValues}]{}{ S }{}}
    & \abstate\
      \begin{array}{c}
        [\text{with}\ S = \{ a\ \vert\ \exists\ \mathtt{id}, (a,\mathtt{id}) \in \abstate \}]
      \end{array}\\ 
    \abstate
    & \specarrow{\alabellong[\mathtt{readRegIds}]{}{ S }{}}
    & \abstate\
      \begin{array}{c}
        [\text{with}\ S = \{ \mathit{id}\ \vert\ \exists\ a, (a,\mathtt{id}) \in \abstate \}]
      \end{array}\\ 
    \abstate
    & \specarrow{{\tt clearReg}(S)}
    & \{ (a',\mathit{id}') \vert (a',\mathit{id}') \in \abstate, \mathit{id}' \notin S \}
  \end{array}
\] 

Here we use $\neg (\mathit{id} \leq \abstate)$ to represents that $\mathtt{id}$ is not less or equal than any identifier of $\abstate$. Method $\alabelshort[{\tt write}]{a,id}$ puts $\{ (a,id) \}$ into the abstract state and then removes from the abstract state all pairs with a less identifier. Method ${\tt readRegValues}()\Rightarrow S$ returns the value of the abstract state. Method ${\tt readRegIds}()\Rightarrow S$ returns the identifiers of the abstract state. Method ${\tt clearReg}(S)$ takes a set $S$ of identifiers as arguments, and removes from $\abstate$ the pairs with identifiers in $S$.


\noindent {\bf Specification of Map}: Each abstract state $\abstate$ is a set of tuples $(\mathit{key},\mathit{value})$, where $\mathit{key}$ is a string and $\mathit{value}$ is a data. Moreover, we assume that each $\mathit{key}$ is unique in $\abstate$. The sequential specification $\specMap$ of map is defined by:

\[
  \begin{array}{rcl}
    \abstate
    & \specarrow{\alabellong[\mathtt{readMap}]{}{ \abstate }{}}
    & \abstate\\ 
    \abstate
    & \specarrow{{\tt clearMap}(S)}
    & \abstate \setminus S\\ 
    \abstate
    & \specarrow{\alabellong[\mathtt{readMapValues}]{}{ S }{}}
    & \abstate\
      \begin{array}{c}
        [\text{with}\ S = \{ \mathit{value}\ \vert\ \exists\ \mathtt{key}, (\mathit{key},\mathtt{value}) \in \abstate \}] 
      \end{array}\\ 
    \abstate
    & \specarrow{\alabellong[\mathtt{readMapKeys}]{}{ S }{}}
    & \abstate\
      \begin{array}{c}
        [\text{with}\ S = \{ \mathit{key}\ \vert\ \exists\ \mathtt{value}, (\mathit{key},\mathtt{value}) \in \abstate \}]
      \end{array}\\
    \abstate
    & \specarrow{{\tt removeMap}(\mathit{key})}
    & \{ (\mathit{key}',\mathit{value}') \vert (\mathit{key}',\mathit{value}') \in \abstate, \mathit{key}' \neq \mathit{key} \}\\
    \big(\ \abstate\ |\ \mathit{key} \notin \abstate \big)
             & \specarrow{ \alabelshort[{\tt addMap}]{\mathit{key},\mathit{value}} }
    & \abstate \cup \{ (\mathit{key},\mathtt{value}) \}
  \end{array}
\]

Here we use $\mathit{key} \notin \abstate$ to represents that for each $(\mathit{key}',\mathit{value}') \in \abstate$, we have $\mathit{key} \neq \mathit{key}'$. Method ${\tt readMap}\Rightarrow S$ returns the pairs of the abstract state. Method $\mathit{clearMap}(S)$ removes pairs of $S$ from the abstract state. Method ${\tt readMapValues}()\Rightarrow S$ returns the set $S$ of values of the abstract state. Method ${\tt readMapKeys}()\Rightarrow S$ returns the set $S$ of keys of the abstract state. Method ${\tt removeMap}(\mathit{key}_1)$ removes the pair of key $\mathit{key}_1$ in the abstract state. Method ${\tt addMap}(\mathit{key},\mathit{value})$ adds the pair $(\mathit{key},\mathit{value})$ into the abstract state. 



\noindent {\bf Specification of List}: %We use the sequential specification of list of our previous work \cite{RA-lin}. 
Each abstract state $\abstate = (l,T)$ contains a sequence $l$ of elements of a given type and a set $T$ of elements appearing in the list. The element $l$ is the list of all input values, whether already removed or not; while $T$ stores the removed values and is used as \emph{tombstone set}. The sequential specification $\specRGA$ of list with add-after interface is defined by: 
  \[\small
    \begin{array}{rcl}
      (l,T)
      & \specarrow{{\tt readList}()\Rightarrow (l/T)}
      & (l,T)\\
      (l,T)
      & \specarrow{{\tt clearList}(S)}
      & (l,T \cup S)\\ 
      \big(\ (l_1 \cdot b \cdot l_2,T\big)\ |\ a\text{ fresh}\ \big)
      & \specarrow{\alabelshort[\mathtt{addAfter}]{b,a}}
      & (l_1 \cdot b \cdot a \cdot l_2,T)\\
      (l,T)
      & \specarrow{{\tt readListLength}\Rightarrow \vert (l/T) \vert}
      & (l,T)\\ 
      (l,T)
      & \specarrow{{\tt removeList}(b)}
      & (l,T \cup \{ b \}) 
 \end{array}
\]
where we denote by $l/T$ the list resulting from removing all elements of $T$ from $l$. The method ${\tt readList}()\Rightarrow l_1$ returns the list content excluding any element appearing in $T$. Method ${\tt clearList}(S)$ adds elements of $S$ into $T$, hence removing elements of $S$ from the list for subsequent calls to the ${\tt read}$ method. Method ${\tt addAfter}(b,a)$ puts $a$ immediately after $b$ in $l$, assuming that each value is put into list at most once. Method ${\tt readListLength}()\Rightarrow i$ returns the length of list content excluding any element appearing in $T$. Method ${\tt removeList}(b)$ adds $b$ into $T$, hence removing $b$ from the list for subsequent calls to the ${\tt read}$ method. 




