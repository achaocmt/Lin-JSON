%!TEX root = draft.tex




\section{Implementation of JSON}
\label{sec:implementation of json}

\forget{
Given a JSON object $o$, the semantics of $o$ is given as a labeled transition system (LTS) $\llbracket o \rrbracket = (\globalstates,\acts,\aglobalstate_0,\rightarrow)$, where $\globalstates$ is a set of global configurations, $\acts$ is the set of transition labels, $\aglobalstate_0$ is the initial configuration, and $\rightarrow \subseteq \globalstates \times \acts\times \globalstates$ is the transition relation. We use the action $\src{\arep}{\alabel}$ to label the generator of $\alabel$ when executed at replica $\arep$, and $\dwn{\arep}{\alabel}$ to label the effector of $\alabel$ when executed at $\arep$. For readability, we use $\aglobalstate \xrightarrow{\aact} \aglobalstate'$ to denote a transition $(\aglobalstate,\aact,\aglobalstate')\in\,\rightarrow$.

A global configuration $(\gstates, \avisord, \downstreams)$ is defined as follows:

\begin{itemize}
\setlength{\itemsep}{0.5pt}
\item[-] $\gstates \in [\reps \rightarrow \localstates]$ maps each replica id into its local configuration. A local configuration $(LO, LS, LV, LL)$ contains the following information

    \begin{itemize}
    \setlength{\itemsep}{0.5pt}
    \item[-] $LO \subseteq \powerset{\objids \times \types}$ contains object identifiers and their types of a replica,

    \item[-] $LS: \objids \rightarrow \states$ contains the replica state of each object of a replica,

    \item[-] $LV: \vars \rightarrow \objids$ maps each variable into a object,

    \item[-] $LL \subseteq \labels$ is the set of operations which has been applied in a replica,
    \end{itemize}

\item[-] The relation $\avisord\subseteq \powerset{\labels \times \labels}$ is the \emph{visibility} relation between operations.

\item[-] $\downstreams\in [\labels \rightarrow \Delta]$ associates to each operation label $\alabel \in LL$ an effector $\effector\in [\states \rightarrow \states]$ ($\Delta$ denotes the set of effectors).
\end{itemize}

Let $id_d$ be the object identifier for cursor $doc$. A method {\tt C2B} is defined in \figureautorefname~\ref{lst:method C2B}. $\alabellong[{\tt C2B}]{EXPR,LS,LV}{bid}{}$ is used to find the object identifier $bid$ of cursor $EXPR$.

% \gpnote[inline, nomargin]{Add an intuitive notion of history.}
% (adapted from~\cite{ShapiroPBZ11})
\begin{figure}[t]
\begin{lstlisting}[basicstyle=\ttfamily\scriptsize,caption={\vspace{-1mm}Method {\tt C2B}.},captionpos=b,label={lst:method C2B}]
  C2B(@|$EXPR$|@, @|$LS$|@, @|$LV$|@):
    assume that @|$EXPR = a_1 \cdot \ldots \cdot a_n$|@
    bid = some fixed initial value
    for(i=1;i<=k;i++)
      if(@|$a_i = obj$|@)
        bid = @|$id_d$|@
      else if(@|$a_i \in \vars$|@)
        bid = @|$LV(a_i)$|@
      else if(type(bid)=map @|$\wedge$|@ @|$a_i$|@=get(key) @|$\wedge$|@ key @|$\in$|@ @|$LS(bid).\alabelshort[{\tt keys}]{}$|@)
        bid = @|$LS(bid).\alabelshort[{\tt get}]{key}$|@
      else if(type(bid)=list @|$\wedge$|@ @|$a_i$|@=idx(j) @|$\wedge$|@ @|$LS(bid).\alabellong[{\tt read}]{}{b_1 \cdot \ldots \cdot b_n}{}$|@ @|$\wedge$|@ @|$j \leq n$|@)
        bid = @|$b_j$|@
      else if(type(bid)=list @|$\wedge$|@ @|$a_i$|@=idx(j) @|$\wedge$|@ @|$LS(bid).\alabellong[{\tt read}]{}{b_1 \cdot \ldots \cdot b_n}{}$|@ @|$\wedge$|@ @|$j > n$|@)
        bid = @|$b_n$|@
    return bid
\end{lstlisting}
\vspace{-5mm}
\end{figure}

A method {\tt findPar} is defined in \figureautorefname~\ref{lst:method findPar}. Given a $bid$ which is an element of a list, $\alabellong[{\tt findPar}]{bid,LO,LS}{bid'}{}$ is used to find the object identifier $bid'$ of the list of $bid$.

% \gpnote[inline, nomargin]{Add an intuitive notion of history.}
% (adapted from~\cite{ShapiroPBZ11})
\begin{figure}[t]
\begin{lstlisting}[basicstyle=\ttfamily\scriptsize,caption={\vspace{-1mm}Method {\tt findPar}.},captionpos=b,label={lst:method findPar}]
  findPar(bid, @|$LO$|@, @|$LS$|@):
    for each @|$(bid',t') \in LO$|@
      if(@|$t' = list$|@)
        assume that @|$LS(bid').\alabellong[{\tt read}]{}{a_1 \cdot \ldots \cdot a_n}{}$|@
          if @|$\exists$|@ j, such that @|$a_j = bid$|@
            return @|$bid'$|@
\end{lstlisting}
\vspace{-5mm}
\end{figure}

The transition relation between global configurations is defined as follows:

\begin{itemize}
\setlength{\itemsep}{0.5pt}
\item[-] When replica $\arep$ do let $\avar = EXPR$:
\end{itemize}

\[
  \inferrule[]
  {\gstates(\arep) = (LO, LS, LV, LL) \\ \alabelshort[{\tt C2B}]{EXPR,LS,LV} = \aobjid}
  {(\gstates, \avisord, \downstreams) \xrightarrow{} (\gstates[\arep \leftarrow (LO, LS, LV[x \leftarrow \aobjid], LL)], \avisord, \downstreams)}
\]

\begin{itemize}
\setlength{\itemsep}{0.5pt}
\item[-] When replica $\arep$ do $EXPR = v$ and $EXPR$ is a register:
\end{itemize}

\[
  \inferrule[]
  {\gstates(\arep) = (LO, LS, LV, LL) \\ \alabelshort[{\tt C2B}]{EXPR,LS,LV} = \aobjid \\ type(\aobjid)=reg  \\ v \neq \{ \} \\ v \neq [ ] \\ \atsource(LS(\aobjid),{\tt write},v) = (\bot,\effector,\ats) \\  \effector(LS(\aobjid)) = \astate' \\ \alabel =  \aobjid.{\ensuremath{{\tt write}(v) \overset{i,\ats}{\Rightarrow}\bot}} \\ \mathit{unique}(i) \\
  \ats\neq\bot\implies (\,\forall \alabel'\in\alabelset.\ \tsof(\alabel') < \ats\,) \\
  \forall \alabel'\in \labeldom{\avisord}.\ \tsof(\alabel') \neq \ats}
  {(\gstates, \avisord, \downstreams) \xrightarrow{\src{\arep}{\alabel}} (\gstates[\arep \leftarrow (LO, LS[\aobjid \leftarrow \astate'],LV,LL \cup \{ \alabel \} ) ], %\alabel
    \avisord \cup (LL \times \{\alabel\}), \downstreams[\alabel \leftarrow \effector])}
\]

In other semantics rules, when a new operation $\alabel$ is generated with identifier $i$ and timestamp $\ats$, the requirement $\mathit{unique}(i)$, $\ats\neq\bot\implies (\,\forall \alabel'\in\alabelset.\ \tsof(\alabel') < \ats\,)$ and $\forall \alabel'\in \labeldom{\avisord}.\ \tsof(\alabel') \neq \ats$ are also used. For brevity, we do not write them explicitly from now on.

\begin{itemize}
\setlength{\itemsep}{0.5pt}
\item[-] When replica $\arep$ do $EXPR = v$, $EXPR$ is a map and $v = \{ \}$:
\end{itemize}

\[
  \inferrule[]
  {\gstates(\arep) = (LO, LS, LV, LL) \\ \alabelshort[{\tt C2B}]{EXPR,LS,LV} = \aobjid \\ type(\aobjid)=map  \\ v = \{ \} \\ \atsource(LS(\aobjid),{\tt clear},\bot) = (\bot,\effector,\ats) \\  \effector(LS(\aobjid)) = \astate' \\ \alabel =  \aobjid.{\ensuremath{{\tt clear}() \overset{i,\ats}{\Rightarrow}\bot}} }
  {(\gstates, \avisord, \downstreams) \xrightarrow{\src{\arep}{\alabel}} (\gstates[\arep \leftarrow (LO, LS[\aobjid \leftarrow \astate'],LV,LL \cup \{ \alabel \} ) ], %\alabel
    \avisord \cup (LL \times \{\alabel\}), \downstreams[\alabel \leftarrow \effector])}
\]

\begin{itemize}
\setlength{\itemsep}{0.5pt}
\item[-] When replica $\arep$ do $EXPR = v$, $EXPR$ is a list and $v = [ ]$:
\end{itemize}

\[
  \inferrule[]
  {\gstates(\arep) = (LO, LS, LV, LL) \\ \alabelshort[{\tt C2B}]{EXPR,LS,LV} = \aobjid \\ type(\aobjid)=list  \\ v = [ ] \\ \atsource(LS(\aobjid),{\tt clear},\bot) = (\bot,\effector,\ats) \\  \effector(LS(\aobjid)) = \astate' \\ \alabel =  \aobjid.{\ensuremath{{\tt clear}() \overset{i,\ats}{\Rightarrow}\bot}} }
  {(\gstates, \avisord, \downstreams) \xrightarrow{\src{\arep}{\alabel}} (\gstates[\arep \leftarrow (LO, LS[\aobjid \leftarrow \astate'],LV,LL \cup \{ \alabel \} ) ], %\alabel
    \avisord \cup (LL \times \{\alabel\}), \downstreams[\alabel \leftarrow \effector])}
\]

\begin{itemize}
\setlength{\itemsep}{0.5pt}
\item[-] When replica $\arep$ do $EXPR.\alabelshort[{\tt addAfter}]{v}$, and $v = \{ \}$: Let $\astate_{m0}$ be the initial replica state of map object.
\end{itemize}

\[
  \inferrule[]
  {\gstates(\arep) = (LO, LS, LV, LL) \\ \alabelshort[{\tt C2B}]{EXPR,LS,LV} = \aobjid_1 \\ \alabellong[{\tt findPar}]{\aobjid_1,LO,LS}{\aobjid_2}{} \\ \atsource(LS(\aobjid_2),{\tt addAfter},\aobjid_1) = (\bot,\effector,\ats) \\ \effector(LS(\aobjid_2)) = \astate' \\ \aobjid_3 = genBID() \\ \alabel =  \aobjid_2.{\ensuremath{{\tt addAfter}(\aobjid_1,\aobjid_3) \overset{i,\ats}{\Rightarrow}\bot}} \\ v = \{ \} }
  {(\gstates, \avisord, \downstreams) \xrightarrow{\src{\arep}{\alabel}} (\gstates[\arep \leftarrow (LO \cup \{ (\aobjid_3,map) \}, LS[\aobjid_2 \leftarrow \astate',\aobjid_3 \leftarrow \astate_{m0}],LV,LL \cup \{ \alabel \} ) ], %\alabel
    \\ \avisord \cup (LL \times \{\alabel\}), \downstreams[\alabel \leftarrow \effector])}
\]

\begin{itemize}
\setlength{\itemsep}{0.5pt}
\item[-] When replica $\arep$ do $EXPR.\alabelshort[{\tt addAfter}]{v}$, and $v = [ ]$: Let $\astate_{l0}$ be the initial replica state of list object.
\end{itemize}

\[
  \inferrule[]
  {\gstates(\arep) = (LO, LS, LV, LL) \\ \alabelshort[{\tt C2B}]{EXPR,LS,LV} = \aobjid_1 \\ \alabellong[{\tt findPar}]{\aobjid_1,LO,LS}{\aobjid_2}{} \\ \atsource(LS(\aobjid_2),{\tt addAfter},\aobjid_1) = (\bot,\effector,\ats) \\ \effector(LS(\aobjid_2)) = \astate' \\ \aobjid_3 = genBID() \\ \alabel =  \aobjid_2.{\ensuremath{{\tt addAfter}(\aobjid_1,\aobjid_3) \overset{i,\ats}{\Rightarrow}\bot}} \\ v = [ ] }
  {(\gstates, \avisord, \downstreams) \xrightarrow{\src{\arep}{\alabel}} (\gstates[\arep \leftarrow (LO \cup \{ (\aobjid_3,list) \}, LS[\aobjid_2 \leftarrow \astate',\aobjid_3 \leftarrow \astate_{l0}],LV,LL \cup \{ \alabel \} ) ], %\alabel
    \\ \avisord \cup (LL \times \{\alabel\}), \downstreams[\alabel \leftarrow \effector])}
\]

\begin{itemize}
\setlength{\itemsep}{0.5pt}
\item[-] When replica $\arep$ do $EXPR.\alabelshort[{\tt addAfter}]{v}$, and $v \neq \{ \} \wedge v \neq [ ]$: Let $\astate_{r0}$ be the initial replica state of register object.
\end{itemize}

\[
  \inferrule[]
  {\gstates(\arep) = (LO, LS, LV, LL) \\ \alabelshort[{\tt C2B}]{EXPR,LS,LV} = \aobjid_1 \\ \alabellong[{\tt findPar}]{\aobjid_1,LO,LS}{\aobjid_2}{} \\ \atsource(LS(\aobjid_2),{\tt addAfter},\aobjid_1) = (\bot,\effector_1,\ats_1) \\ \effector_1(LS(\aobjid_2)) = \astate_1 \\ \aobjid_3 = genBID() \\ \alabel_1 =  \aobjid_2.{\ensuremath{{\tt addAfter}(\aobjid_1,\aobjid_3) \overset{i_1,\ats_1}{\Rightarrow}\bot}} \\ v \neq \{ \} \\ v \neq [ ] \\ \atsource(LS(\astate_{r0},{\tt write},v) = (\bot,\effector_2,\ats_2) \\ \effector_2(\astate_{r0}) = \astate_2 \\ \alabel_2 =  \aobjid_3.{\ensuremath{{\tt write}(v) \overset{i_2,\ats_2}{\Rightarrow}\bot}} }
  {(\gstates, \avisord, \downstreams) \xrightarrow{\src{\arep}{\alabel_1,\alabel_2}} (\gstates[\arep \leftarrow (LO \cup \{ (\aobjid_3,reg) \}, LS[\aobjid_2 \leftarrow \astate_1,\aobjid_3 \leftarrow \astate_2],LV,LL \cup \{ \alabel_1,\alabel_2 \} ) ], %\alabel
    \\ \avisord \cup (LL \times \{\alabel_1, \alabel_2\}) \cup \{ (\alabel_1,\alabel_2) \}, \downstreams[\alabel_1 \leftarrow \effector_1, \alabel_2 \leftarrow \effector_2])}
\]

\begin{itemize}
\setlength{\itemsep}{0.5pt}
\item[-] When replica $\arep$ do $EXPR.delete$ and $EXPR$ is an element of a list:
\end{itemize}

\[
  \inferrule[]
  {\gstates(\arep) = (LO, LS, LV, LL) \\ \alabelshort[{\tt C2B}]{EXPR,LS,LV} = \aobjid_1 \\ \alabellong[{\tt findPar}]{\aobjid_1,LO,LS}{\aobjid_2}{} \\ \atsource(LS(\aobjid_2),{\tt delete},\aobjid_1) = (\bot,\effector,\ats) \\ \effector(LS(\aobjid_2)) = \astate' \\ \alabel =  \aobjid_2.{\ensuremath{{\tt delete}(\aobjid_1) \overset{i,\ats}{\Rightarrow}\bot}} }
  {(\gstates, \avisord, \downstreams) \xrightarrow{\src{\arep}{\alabel}} (\gstates[\arep \leftarrow (LO , LS[\aobjid_2 \leftarrow \astate'],LV,LL \cup \{ \alabel \} ) ], %\alabel
    \avisord \cup (LL \times \{\alabel\}), \downstreams[\alabel \leftarrow \effector])}
\]

\begin{itemize}
\setlength{\itemsep}{0.5pt}
\item[-] When replica $\arep$ do $EXPR.delete$ and $EXPR$ is an element of a map:
\end{itemize}

\[
  \inferrule[]
  {\gstates(\arep) = (LO, LS, LV, LL) \\ \alabelshort[{\tt C2B}]{EXPR,LS,LV} = \aobjid_1 \\ \exists \aobjid_2,key, such \ that (type(\aobjid_2) = map \wedge LS(\aobjid_2).{\tt get}(key) = \aobjid_1) \\ \atsource(LS(\aobjid_2),{\tt remove},key) = (\bot,\effector,\ats) \\ \effector(LS(\aobjid_2)) = \astate' \\ \alabel =  \aobjid_2.{\ensuremath{{\tt remove}(key) \overset{i,\ats}{\Rightarrow}\bot}} }
  {(\gstates, \avisord, \downstreams) \xrightarrow{\src{\arep}{\alabel}} (\gstates[\arep \leftarrow (LO , LS[\aobjid_2 \leftarrow \astate'],LV,LL \cup \{ \alabel \} ) ], %\alabel
    \avisord \cup (LL \times \{\alabel\}), \downstreams[\alabel \leftarrow \effector])}
\]

\begin{itemize}
\setlength{\itemsep}{0.5pt}
\item[-] When replica $\arep$ do $EXPR.{\tt create}(key,\atype)$: Let $\astate_{\atype 0}$ be the initial replica state of type $\atype$.
\end{itemize}

\[
  \inferrule[]
  {\gstates(\arep) = (LO, LS, LV, LL) \\ \alabelshort[{\tt C2B}]{EXPR,LS,LV} = \aobjid_1 \\ type(\aobjid_1) = map \\ \aobjid_2 = getBID() \\ \atsource(LS(\aobjid_1),{\tt add},key,\aobjid_2) = (\bot,\effector,\ats) \\ \effector(LS(\aobjid_1)) = \astate' \\ \aobjid_3 = genBID() \\ \alabel =  \aobjid_1.{\ensuremath{{\tt add}(key,\aobjid_2) \overset{i,\ats}{\Rightarrow}\bot}} }
  {(\gstates, \avisord, \downstreams) \xrightarrow{\src{\arep}{\alabel}} (\gstates[\arep \leftarrow (LO \cup \{ (\aobjid_2,\atype) \}, LS[\aobjid_1 \leftarrow \astate',\aobjid_2 \leftarrow \astate_{\atype 0}],LV,LL \cup \{ \alabel \} ) ], %\alabel
    \\ \avisord \cup (LL \times \{\alabel\}), \downstreams[\alabel \leftarrow \effector])}
\]

\begin{itemize}
\setlength{\itemsep}{0.5pt}
\item[-] When replica $\arep$ apply an effector:
\end{itemize}

\[
  \inferrule[\text{\sc DownStream}]
  {\gstates(\arep) = (LO, LS, LV, LL) \\ \alabel \in \mathsf{min}_{\avisord}(\labeldom{\avisord} \setminus LL) \\
    \downstreams(\alabel)= \delta \\ obj(\alabel) = \aobjid_1, \delta(LS(\aobjid_1)) = \astate_1, (\aobjid_2,\atype_2) = getObj(\alabel)}
  {(\gstates, \avisord, \downstreams) \xrightarrow{\dwn{\arep}{\alabel}} (\gstates[\arep \leftarrow (LO \cup \{ (\aobjid_2,\atype_2) \}, LS[\aobjid_1 \leftarrow \astate_1, \aobjid_2 \leftarrow \astate_{\atype_2 0}],LV,LL \cup \{ \alabel \} ) ], %\alabel
    \\ \avisord \cup (LL \times \{\alabel\}), \downstreams[\alabel \leftarrow \effector])}
\]

Here ${\tt getObj(\alabel)}$ works as follows: If $\alabel = \_.\alabelshort[{\tt addAfter}]{a,b}$ and $b$ is of type $\atype$, then returns $(b,\atype)$; else, if $\alabel = \_.\alabelshort[{\tt add}]{key,a}$ and $a$ is of type $\atype$, then returns $(b,\atype)$; else, returns $\emptyset$.

\begin{itemize}
\setlength{\itemsep}{0.5pt}
\item[-] When replica $\arep$ do $yield$:
\end{itemize}

\[
  \inferrule[]
  {(\gstates_0, \avisord_0, \downstreams_0) \xrightarrow{\dwn{\arep}{\alabel_1}} (\gstates_1, \avisord_1, \downstreams_1) \ldots \xrightarrow{\dwn{\arep}{\alabel_k}} (\gstates_k, \avisord_k, \downstreams_k)}
  {(\gstates_0, \avisord_0, \downstreams_0) \xrightarrow{yield} (\gstates_k, \avisord_k, \downstreams_k)}
\]

\begin{itemize}
\setlength{\itemsep}{0.5pt}
\item[-] When replica $\arep$ do $EXPR.keys$ and $EXPR$ is a map:
\end{itemize}

\[
  \inferrule[]
  {\gstates(\arep) = (LO, LS, LV, LL) \\ \alabelshort[{\tt C2B}]{EXPR,LS,LV} = \aobjid \\ type(\aobjid) = map \\ LS(\aobjid).\alabellong[{\tt keys}]{}{S_1}{} }
  {(\gstates, \avisord, \downstreams) \xrightarrow{\arep.EXPR.keys \Rightarrow S_1} (\gstates, \avisord, \downstreams)}
\]

\begin{itemize}
\setlength{\itemsep}{0.5pt}
\item[-] When replica $\arep$ do $EXPR.length$ and $EXPR$ is a list:
\end{itemize}

\[
  \inferrule[]
  {\gstates(\arep) = (LO, LS, LV, LL) \\ \alabelshort[{\tt C2B}]{EXPR,LS,LV} = \aobjid \\ type(\aobjid) = list \\ LS(\aobjid).\alabellong[{\tt length}]{}{j}{} }
  {(\gstates, \avisord, \downstreams) \xrightarrow{\arep.EXPR.legnth \Rightarrow j} (\gstates, \avisord, \downstreams)}
\]

\begin{itemize}
\setlength{\itemsep}{0.5pt}
\item[-] When replica $\arep$ do $EXPR.values$ and $EXPR$ is a register:
\end{itemize}

\[
  \inferrule[]
  {\gstates(\arep) = (LO, LS, LV, LL) \\ \alabelshort[{\tt C2B}]{EXPR,LS,LV} = \aobjid \\ type(\aobjid) = reg \\ LS(\aobjid).\alabellong[{\tt values}]{}{S_1}{} }
  {(\gstates, \avisord, \downstreams) \xrightarrow{\arep.EXPR.values \Rightarrow S_1} (\gstates, \avisord, \downstreams)}
\]

\begin{itemize}
\setlength{\itemsep}{0.5pt}
\item[-] When replica $\arep$ do $EXPR.type$:
\end{itemize}

\[
  \inferrule[]
  {\gstates(\arep) = (LO, LS, LV, LL) \\ \alabelshort[{\tt C2B}]{EXPR,LS,LV} = \aobjid \\ \exists \atype, such \ that \ (\aobjid,\atype) \in LO }
  {(\gstates, \avisord, \downstreams) \xrightarrow{\arep.EXPR.type \Rightarrow \atype} (\gstates, \avisord, \downstreams)}
\]
}
